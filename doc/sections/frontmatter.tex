\pagenumbering{gobble}
\begin{abstract}
    The purpose of this library is to provide tools for the study of the most beautiful discipline of mathematics: geometry and geometric analysis. In doing so, we restrict ourselves to 2-manifolds. For a mathematician, this may initially seem like a major restriction. However, it allows us, in a relatively simple way, to represent 2-manifolds as ''meshes`` and to develop powerful tools to study them. 
    
    The abstraction hardly needs to be restricted at all, because the proposed calculus for meshes makes it possible to develop new geometries with comparatively little effort. Properties of these meshes can then be examined using \maniflow. For example, we provide tools to break down meshes into their connected components. You can also use \maniflow{} to determine the orientability of a mesh. It is also possible to run a geometric flow, such as the mean curvature flow, on a mesh. This means that \maniflow{} can also be used to examine meshes with regard to curvature (Gaussian curvature, mean curvature).
    
    \maniflow{} also provides the option of creating images of the meshes. This makes it possible, for example, to create animations of geometric flows etc.
\end{abstract}
\tableofcontents
\listoffigures
\vfill
\hrule\vspace*{0.2cm}
\noindent\footnotesize{\maniflow{} was developed as part of the course M.Mat.0731 ''Advanced practical course in scientific computing`` at Georg-August University Göttingen.

The image on the frontpage is taken from \href{https://unsplash.com/de/fotos/blaues-und-rotes-licht-digitales-hintergrundbild-wxj729MaPRY}{\url{https://unsplash.com/de/fotos/blaues-und-rotes-licht-digitales-hintergrundbild-wxj729MaPRY}}
\restoregeometry
\normalsize
\pagenumbering{arabic}